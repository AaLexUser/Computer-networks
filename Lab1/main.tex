\documentclass[12pt,onecolumn]{article}
\usepackage[utf8]{inputenc} % UTF8 input encoding
\usepackage[T2A]{fontenc}   % T2A font encoding for Cyrillic script
\usepackage[russian]{babel} % Russian language support
\usepackage{listings}
\usepackage{float}
\usepackage{mathtools}
\everymath{\displaystyle}
\usepackage{listings} 
\usepackage[usenames]{color}
\usepackage{geometry}
\usepackage{verbatim}
\usepackage{tabularray}
\usepackage{color}
\newcommand{\nparagraph}[1]{\paragraph{#1}\mbox{}\\}
\geometry{
  a4paper,
  top=20mm, 
  right=25mm, 
  bottom=20mm, 
  left=20mm
}

\begin{document}
\setcounter{tocdepth}{4}
\begin{center}
    Федеральное государственное автономное образовательное учреждение высшего образования "Национальный Исследовательский Университет ИТМО"\\ 
    Мегафакультет Компьютерных Технологий и Управления\\
    Факультет Программной Инженерии и Компьютерной Техники \\
    \includegraphics[scale=0.3]{image/itmo.jpg} % нужно закинуть картинку логтипа в папку с отчетом
\end{center}
\vspace{1cm}


\begin{center}
    \textbf{Домашнее задание 2}\\
    по дисциплине\\
    \textbf{Компьютерные сети}
\end{center}

\vspace{2cm}

\begin{flushright}
  Выполнил Студент  группы P33102\\
  \textbf{Лапин Алексей Александрович}\\
  Преподаватель: \\
  \textbf{Авксентьева Елена Юрьевна}\\
\end{flushright}

\vspace{6cm}
\begin{center}
    г. Санкт-Петербург\\
    2023г.
\end{center}

\newpage
\tableofcontents
\newpage

\section{Цель работы}
Изучение принципов построения и настройки моделей компьютерных сетей в среде NetEmul.

В процессе выполнения лабораторной работы (ЛР) необходимо:

\begin{enumerate}
    \item построить три простейшие модели компьютерной сети;
    \item выполнить настройку сети, заключающуюся в присвоении IP-адрес интерфейсам сети;
    \item выполнить тестирование разработанных сетей путем проведения экспериментов по передаче данных на основе протокола UDP;
    \item сохранить разработанные модели компьютерных сетей для демонстрации процессов передачи данных при защите лабораторной работы.
\end{enumerate}

\section{Этап 1. Знакомство с NetEmul на примере простейшей сети из двух компьютеров}


\end{document}